\documentclass[aspectratio=169,11pt,hyperref={colorlinks=true}]{beamer}
\usetheme{boxes}
\setbeamertemplate{navigation symbols}{}
\definecolor{openstack}{RGB}{149,0,4}
\setbeamercolor{titlelike}{fg=openstack}
\setbeamercolor{structure}{fg=openstack}
\hypersetup{colorlinks,urlcolor=openstack}
\setbeamertemplate{footline}[frame number]
% Inserting graphics
\usepackage{graphicx}
% Side-by-side figures, etc
\usepackage{subfigure}
% Code snippits
\usepackage{listings}
% Color stuff
\usepackage{color}
\usepackage{amsmath}
\usepackage{tikz}
\usepackage{tipa}
\newcommand\RBox[1]{%
  \tikz\node[draw,rounded corners,align=center,] {#1};%
}
\usepackage{hyperref}
%\usecolortheme{buzz}
%\usecolortheme{wolverine}
%\usetheme{Boadilla}
\usepackage[T1]{fontenc}

\definecolor{mygreen}{rgb}{0,0.6,0}
\definecolor{mygray}{rgb}{0.5,0.5,0.5}
\definecolor{mymauve}{rgb}{0.58,0,0.82}

\lstset{ %
  backgroundcolor=\color{white},   % choose the background color; you must add \usepackage{color} or \usepackage{xcolor}
  breakatwhitespace=false,         % sets if automatic breaks should only happen at whitespace
  breaklines=true,                 % sets automatic line breaking
  captionpos=b,                    % sets the caption-position to bottom
  commentstyle=\color{openstack},  % comment style
  extendedchars=true,              % lets you use non-ASCII characters; for 8-bits encodings only, does not work with UTF-8
  keepspaces=true,                 % keeps spaces in text, useful for keeping indentation of code (possibly needs columns=flexible)
  keywordstyle=\color{blue},       % keyword style
  otherkeywords={*,...},           % if you want to add more keywords to the set
  numbersep=5pt,                   % how far the line-numbers are from the code
  numberstyle=\tiny\color{mygray}, % the style that is used for the line-numbers
  rulecolor=\color{black},         % if not set, the frame-color may be changed on line-breaks within not-black text (e.g. comments (green here))
  showspaces=false,                % show spaces everywhere adding particular underscores; it overrides 'showstringspaces'
  showstringspaces=false,          % underline spaces within strings only
  showtabs=false,                  % show tabs within strings adding particular underscores
  stringstyle=\color{openstack},   % string literal style
}

\setbeamerfont{caption}{series=\normalfont,size=\fontsize{8}{10}}
\setbeamertemplate{caption}{\raggedright\insertcaption\par}

\setlength{\abovecaptionskip}{0pt}
\setlength{\floatsep}{0pt}

\author[Andrea Frittoli]{%
    \texorpdfstring{
        \begin{columns}
            \column{.45\linewidth}
            \centering
            Andrea Frittoli\\
            \href{mailto:andrea.frittoli@gmail.com}{andrea.frittoli@gmail.com}\\
            \texttt{andreaf on Freenode}\\
            \texttt{@blackchip76 on Twitter}
        \end{columns}
   }
   {Andrea Frittoli}
}
\date{Jun 2, 2017}

\title[Testing at Scale
\hspace{2em}\insertframenumber/\inserttotalframenumber]{Testing at Scale}

\begin{document}

{
\setbeamertemplate{background canvas}{\includegraphics[width=\paperwidth,height=\paperheight]{background_title.png}}
\setbeamertemplate{footline}{}
\begin{frame}[noframenumbering]
    \setbeamercolor{titlelike}{fg=white}
    \setbeamercolor{structure}{fg=white}
    \setbeamercolor{normal text}{fg=white}
    \hypersetup{colorlinks,urlcolor=white}
    \setbeamercolor{author}{fg=white}
    \setbeamercolor{date}{fg=white}
    \setbeamercolor{background}{bg=openstack}
    \titlepage{}
    \centering
    \href{https://github.com/andreafrittoli/testing\_at\_scale}{https://github.com/andreafrittoli/testing\_at\_scale}
\end{frame}
}

\begin{frame}
    \frametitle{OpenStack CI in Numbers}
       \begin{columns}
        \column{0.4\linewidth}
    \begin{itemize}
        \item{$\sim$1800 contributors to Pike}
        \item{6 Clouds + 2 run by OpenStack Infra}
        \item{1600+ repos in openstack(|-dev|-infra)}
        \item{20k lines Zuul configuration}
    \end{itemize}
        \column{0.6\linewidth}
    \begin{figure}
    \begin{center}
    	\includegraphics[width=1\textwidth]{zuul_all_jobs.png}
         \caption{Source: Zuul, Graphite}
    \end{center}
    \end{figure}
       \end{columns}
\end{frame}

\begin{frame}
    \frametitle{OpenStack CI in Numbers}
    \begin{figure}
    \begin{center}
    	\includegraphics[width=0.7\textwidth]{daily_count.png}
         \caption{Source: subunit2sql-graph dailycount}
    \end{center}
    \end{figure}
\end{frame}

\note{Describe OpenStack CI today in numbers}

\begin{frame}
    \frametitle{OpenStack QA Mission Statement}
    \begin{center}
    	\begin{quote}{"Develop, maintain, and initiate tools and plans to ensure the upstream stability and quality of OpenStack, and its release readiness at any point during the release cycle."}
	\end{quote}
    \end{center}
\end{frame}


\begin{frame}
    \frametitle{Testing OpenStack}
       \begin{columns}
        \column{0.6\linewidth}
    \begin{figure}
    \begin{center}
    	\includegraphics[width=1\textwidth]{zuul_status.png}
         \caption{Source: \href{http://status.openstack.org/zuul}{http://status.openstack.org/zuul}}
    \end{center}
    \end{figure}
        \column{0.4\linewidth}
    \begin{itemize}
        \item{Release ready master}
        \item{Check \& Gate Pipelines}
        \item{Ephemeral test system}
    \end{itemize}
       \end{columns}
\end{frame}

\note{Quick explanation about how OpenStack CI works}

\begin{frame}
    \frametitle{Beyond the gate}
    \begin{columns}
         \column{0.3\linewidth}
            \begin{itemize}
                \item{Periodic tests}
                \item{Release jobs}
            \end{itemize}
         \column{0.7\linewidth}
            \begin{figure}
                \includegraphics[width=1\textwidth]{periodic_jobs.png}
                \caption{Source: \href{http://status.openstack.org/openstack-health/\#/g/build_queue/periodic}{status.openstack.org/openstack-health/}}
            \end{figure}
    \end{columns}
\end{frame}

\note{Details about extra stuff and how it's hard to get visibility on it}

\begin{frame}
    \frametitle{The unbalanced scale}
    \begin{columns}
         \column{0.3\linewidth}
             \begin{itemize}
                 \item{Core reviewers}
                 \item{Horizontal teams}
                 \item{Running CI}
             \end{itemize}
         \column{0.4\linewidth}
            \begin{figure}
                \includegraphics[width=1\textwidth]{unbalanced_scale.png}
                \caption{Not the scale I meant...}
            \end{figure}
         \column{0.3\linewidth}
             \begin{itemize}
                 \item{Commits}
                 \item{\# of Tests}
                 \item{\# of Configurations}
             \end{itemize}
    \end{columns}
\end{frame}

\note{Complexity grows. Amount of testing grows. Contribution grow, but not in a balanced way.}

\begin{frame}
    \frametitle{Life as a contributor}
    \begin{figure}
        \begin{minipage}[b]{0.45\linewidth}
            \centering
    	   \includegraphics[width=1\textwidth]{jobs_per_change.png}
            \caption{Source: subunit2sql DB}
        \end{minipage}
        \hspace{0.5cm}
        \begin{minipage}[b]{0.45\linewidth}
            \centering
    	   \includegraphics[width=1\textwidth]{patchsets_per_change.png}
            \caption{Source: Gerrit API}
        \end{minipage}
    \end{figure}
\end{frame}

\begin{frame}
    \frametitle{Life as a contributor}
       \begin{columns}
        \column{0.4\linewidth}
    \begin{itemize}
        \item{Worst case}
        \item{Recheck}
    \end{itemize}
        \column{0.6\linewidth}
    \begin{figure}
            \centering
            \includegraphics[width=1\textwidth]{max_job_per_changes.png}
            \caption{Source: subunit2sql DB}
    \end{figure}
       \end{columns}
\end{frame}

\note{How does scale-up impact contributors?}

\begin{frame}
    \frametitle{My day in reviews}
    \begin{figure}
    \begin{center}
         \includegraphics[width=1\textwidth]{pending_reviews.png}
         \caption{Source: gertty}
    \end{center}
    \end{figure}
\end{frame}

\begin{frame}
    \frametitle{My day in reviews}
    \begin{itemize}
        \item{Change in devstack:}
    \end{itemize}
    \begin{figure}
    \lstinputlisting[basicstyle=\tiny,language=bash]{code/devstack_patch.sh}
    \caption{Source: \href{https://review.openstack.org/\#/c/468060/2/lib/tempest@536}{https://review.openstack.org/468060}}
    \end{figure}
    \begin{itemize}
        \item{CI reports all green}
        \item{Can you see the error?}
    \end{itemize}
\end{frame}

\note{How does scale-up impact reviewers?}

\begin{frame}
    \frametitle{Keep the gate flowing}
       \begin{columns}
        \column{0.4\linewidth}
    \begin{itemize}
        \item{Beginning of Pike:}
        \item{Number of services}
        \item{Service memory footprint}
        \item{Test load}
        \item{Libvirt instability under load}
        \item{\href{https://review.openstack.org/\#/c/439638/}{Tempest bug in ssh test code}}
        \item{Vanilla database configuration}
    \end{itemize}
        \column{0.6\linewidth}
    \begin{figure}
    \begin{center}
    	\includegraphics[width=1\textwidth]{failure_rate.png}
         \caption{Source: Zuul, Graphite}
    \end{center}
    \end{figure}
       \end{columns}
\end{frame}

\note{How does scale-up impact gate maintainers?}

\begin{frame}
    \frametitle{Data!}
    \begin{itemize}
        \item{50MB compressed logs for integration test run}
        \begin{itemize}
            \item{node setup}
            \item{system, service and test log files}
            \item{system, service and test configuration}
            \item{system metrics (dstat data)}
            \item{libvirt logs}
        \end{itemize}
        \item{Test result data (Zuul \& subunit)}
        \item{Gerrit event streams}
    \end{itemize}
\end{frame}

\note{Luckily we have data. A lot for one contributor. Too much for reviewers. Unmanageable for gate maintainers.}

\begin{frame}
    \frametitle{Single run view}
    \begin{columns}
         \column{0.3\linewidth}
        \begin{itemize}
            \item{Deeplink into \href{http://logs.openstack.org/22/466822/1/gate/gate-tempest-dsvm-neutron-full-ubuntu-xenial/d6f19c0/logs/}{logs}}
            \item{HTML view of tests}
            \item{Stackviz}
            \item{Ansible Run Analysis}
        \end{itemize}
         \column{0.7\linewidth}
            \begin{figure}
                \includegraphics[width=1\textwidth]{stackviz.png}
                \caption{Source: \href{http://logs.openstack.org/22/466822/1/gate/gate-tempest-dsvm-neutron-full-ubuntu-xenial/d6f19c0/logs/stackviz/\#/stdin/timeline?test=tempest.api.network.test_dhcp_ipv6.NetworksTestDHCPv6.test_dhcpv6_stateless_no_ra}{Stackviz}}
            \end{figure}
    \end{columns}
\end{frame}

\note{Tools for single run data view, and how do they help the different actors}

\begin{frame}
    \frametitle{Runtime variance}
    \begin{figure}
    \begin{center}
    	\includegraphics[width=0.7\textwidth]{runtime_variance.png}
         \caption{Source: subunit2sql-graph run\_time}
    \end{center}
    \end{figure}
\end{frame}

\note{About the volume test and runtime variance across different nodes and clouds}

\begin{frame}
    \frametitle{Deeper into data}
    \begin{itemize}
        \item{ELK, Elastic Recheck \href{http://status.openstack.org/elastic-recheck/}{[DEMO: http://status.openstack.org/elastic-recheck/]}}
        \item{Subunit2sql, OpenStack Health \href{http://status.openstack.org/openstack-health/\#/}{[DEMO: http://status.openstack.org/openstack-health/]}}
    \end{itemize}
\end{frame}

\note{Explain and Demo E-R and O-H}

\begin{frame}
    \frametitle{Other Data Sources}
    \begin{itemize}
        \item{Graphite: \href{http://graphite.openstack.org}{http://graphite.openstack.org}}
        \item{Grafana: \href{http://grafana.openstack.org}{http://grafana.openstack.org}}
        \item{Firehose: \href{https://docs.openstack.org/infra/system-config/firehose.html}{https://docs.openstack.org/infra/system-config/firehose.html}}
    \end{itemize}
\end{frame}

\note{What other kind of data sources we have. Firehose simple demo.}

\begin{frame}
    \frametitle{What's next?}
    \begin{itemize}
        \item{Metrics over time from OS, MySQL, RabbitMQ, etc}
        \item{CRM114}
        \item{Artificial intelligence}
    \end{itemize}
\end{frame}

\note{Things we'd like to work on next.}

\begin{frame}
    \frametitle{Playing with Machine Learning}
    \begin{itemize}
        \item{Long Short Term Memory network, RNN}
        \item{Trained with nova logs from periodic test runs}
    \end{itemize}
    \begin{itemize}
        \item{DEMO}
    \end{itemize}
\end{frame}

\begin{frame}
    \frametitle{AI to the rescue}
    \begin{itemize}
        \item{Make anomalous log lines emerge}
    \end{itemize}
    \begin{itemize}
        \item{Help identitify new elastick recheck signatures}
        \item{Catch changes in behaviour that we do not test for}
        \item{Help review process}
    \end{itemize}
\end{frame}

\begin{frame}
    \frametitle{References}
    \begin{itemize}
        \item{\href{http://status.openstack.org}{http://status.openstack.org}, 
                  \href{http://review.openstack.org}{http://review.openstack.org}}
        \item{\href{https://docs.openstack.org/developer/subunit2sql/graph.html}{https://docs.openstack.org/developer/subunit2sql/graph.html}}
        \item{\href{https://github.com/sherjilozair/char-rnn-tensorflow}{https://github.com/sherjilozair/char-rnn-tensorflow}}
        \item{\href{https://docs.openstack.org/infra/system-config/logstash.html\#crm114}{https://docs.openstack.org/infra/system-config/logstash.html\#crm114}}
        \item{\href{https://github.com/andreafrittoli/testing\_at\_scale/}{https://github.com/andreafrittoli/testing\_at\_scale/}}
    \end{itemize}
    \begin{itemize}
        \item{IRC: \#openstack-qa}
        \item{email: openstack-dev@lists.openstack.org, tag [QA]}
    \end{itemize}
\end{frame}


\section{Questions}
\begin{frame}[c]
    %\frametitle{Questions?}
    \begin{center}
        \Huge Thank you!\\Questions?
    \end{center}
\end{frame}

\end{document}
